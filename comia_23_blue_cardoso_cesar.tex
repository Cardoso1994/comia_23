% Created 2023-03-27 Mon 19:35
% Intended LaTeX compiler: pdflatex
\documentclass[a4paper]{llncs}
\usepackage[hidelinks]{hyperref}
\usepackage[utf8]{inputenc}
\usepackage[T1]{fontenc}
\usepackage{graphicx}
\usepackage{longtable}
\usepackage{wrapfig}
\usepackage{rotating}
\usepackage[normalem]{ulem}
\usepackage{amsmath}
\usepackage{amssymb}
\usepackage{capt-of}
\usepackage{hyperref}
\usepackage{listings}
\usepackage{cite}
\usepackage{graphicx}
\usepackage{cleveref}
\usepackage{multirow}
\usepackage{graphicx}
\usepackage[spanish, american]{babel}
\usepackage[]{inputenc}
\voffset=10mm
\hoffset=-3.5mm
\widowpenalty=9999
\clubpenalty=9999
\hyphenpenalty=500
\exhyphenpenalty=500
\pagestyle{empty}
\date{\today}
\title{Paper Comia 23}
\begin{document}




\title{Pseudoetiquetado para el análisis de polaridad en tuits: un primer
    acercamiento}
  % \author{}
  % \institute{}
  \author{Diana Jimenez, Marco A. Cardoso-Moreno, Cesar Macias}
  \authorrunning{Jimenez et al.}


  \institute{Instituto Politécnico Nacional, \\ Centro de Investigación en Computación, \\
    Laboratorio de Ciencias Cognitivas Computacionales, Ciudad de México, México \\
  \email{correo_blue@cic.ipn.mx, \{correo_blue,mcardosom2021,correo_cesar\}@cic.ipn.mx}}

\maketitle


\begin{abstract}



\keywords{BCI \and EEG \and ERP \and P300}
\end{abstract}

\section{Introducción}
\label{sec:orgb0ab483}
Las redes sociales hoy en día forman una parte de la vida cotidiana para la
población en general, ya sea para cuestiones de relaciones interpersonales,
\emph{networking} e incluso, para la consulta y diseminación de información
\cite{improving_sentiment_prediction,tesis_cesar,jtaer18020039}. A partir de
este incremento en el uso de redes sociales, intensificado en años recientes
gracias a la pandemia de COVID-19 \cite{greenhow2021inquiring}, es que estas
plataformas se han vuelto parte del discurso público, ya que los algoritmos
utilizados en las mismas permiten a sus usuarios interactuar con diversos grupos
sociales, lo que los mantiene al tanto de los eventos y problemáticas actuales
\cite{BASTICK2021would_you_notice}.

En particular, Twitter no presenta muchas restricciones sobre el contenido de
las  publicaciones que sus usuarios pueden efectuar por lo que, en general,
suelen ser sobre cualquier tema, esta aparente libertad que la plataforma provee
es la principal razón de que esta red social tiene preferencia entre los
internautas para, en ella, mostrar sus opiniones \cite{tesis_cesar}.

Dentro de las áreas de procesamiento de lenguaje natural (PLN) y lingüística
computacional existe la tarea de análisis de opiniones, que consiste en,
mediante el análisis del texto donde un comentario opinión fue expresado,
determinar el la opinión que una persona sobre el tema en cuestión
\cite{mejova2009sentiment}; el análisis de la polaridad en una opinión se
considera, a su vez, una subtarea de este campo \cite{gambino-2019}. Determinar
la polaridad de un texto se refiere, entonces, a clasificar, dado un texto, si
la opinión que se ha vertido en este es positiva o negatica, es decir, qué tan
polarizada es.

\section{Revisión de la literatura}
\label{sec:org158fde6}
La clasificación de textos, por su parte, puede llevarse a cabo mediante
estrategias de apredizaje automática, específicamente, aprendizaje supervisado.
Estas técnicas han sido, y siguen siendo, ampliamente utilizadas en la
clasificación de textos para diferentes tareas, siendo una de las más destacadas
el análisis de sentimientos. Por ejemplo, en \cite{jaca2023sentiment} se hace uso
de redes neuronales recurrentes (RNR), específicamente redes Bi-LSTM (del inglés
\emph{Bi Long Short-Term Memory}) para esta tarea;  de manera similar,
\cite{depression_detection} utilizan Twitter como un medio donde la gente puede
expresar síntomas de depresión que requieren ser reportados por un individuo con
esta afectación psicólogica para detectar dicho padecimiento de manera temprana,
para lo cual utilizaron RNR tradicionales, así como redes LSTM; en
\cite{roberta_gru} utilizan el modelo de transforme RoBERTa-GRU (del inglés
\emph{Robustly Optimized BERT Pretraining Approach} y \emph{Gated Recurrent Units}) para
la clasificación de sentimientos en diversos datasets considerados como
\emph{baselines}; por su parte, \cite{sentiment_naive_bayes} utilizan el clasificador
Naive-Bayes para la misma tarea, apoyandose del recurso léxico \emph{sentiwordnet}
para agregar a cada palabra un puntaje de sentimiento positivo, negativo u
objetivo.

En cuanto a la tarea específica de análisis de polaridad en texto, uno de los
primeros trabajos que se llevaron a cabo fue aquel de
\cite{pang_sentiment_classification}, en el cual se utilizaron clasificadores
tradicionales, tales como: Naive Bayes, Entropía Máxima y Máquinas de Soporte
Vectorial (SVM, del inglés \emph{Support Vector Machines}) para la clasificación de
polaridad de reseñas (en inglés) de películas; en \cite{scope_of_negation_camara}
se realiza un estudio sobre el impacto de la negación en la clasificación de la
polaridad en tuits en español, concluyendo que el tomar en cuenta dicho aspecto
contribuye significativamente a una mejora en la clasificación de la polaridad;
además, se encuentra el trabajo de \cite{paper_cesar}, donde se utilizan múltiples
clasificadores, tales como: Entropía Máxima, Naive Bayes Multinomial, SVMs y
BETO, un modelo BERT (del inglés \emph{Bidirectional Encoder Representations from
Transformers}) entrenado con un corpues en español, para obtener la polaridad
de tuits en español, lo que incluía encabezados de noticias, los hilos de la
conversación correspondiente a dichos encabezados, tuits citados y los hilos de
conversación que se generaron a partir de éstos.

Por otro lado, se ha observado que los modelos de aprendizaje de máquina y
aprendizaje profundo suelen ver afectado su desempeño cuando no se cuenta con
suficientes datos, por lo que se suelen utilizar técnicas de aumento de datos y,
en los casos en los que no se puede preservar las etiquetas, pseudoetiquetado,
tal es el caso de \cite{gelbukh-2018-aggression}, donde se utilizó dicha técnica
para mejorar el desmepeño de diversas arquitecturas de redes neuronales para la
tarea de detección de agresión en redes sociales. También destaca el trabajo de
\cite{pseudoetiquedato_transformers}, donde se hace uso del modelo DistilBERT en
la tarea de clasificación de preguntas para su incorporación de sistemas tipo
chatbots dedicados a responder preguntas, para contravenir la falta the datos
etiquetados se utilizan técnicas de pseudoetiquetado, obteniendo como resultado
que el modelo cuyo banco de datos de entrenamiento datos pseudoetiquetados
presentó un mejor desempeño que aquellos con los cuáles no se utilizó dicha
técnica para su entrenamiento. Por último, en \cite{blue} se hace uso del
pseudoetiquetado para la mejora en los sistemas de detección de noticias, ya que
el etiquetado manual de texto suele ser una tarea laboriosa, sobre todo dada la
ingente cantidad de recursos disponibles en internet, lo que resulta en una
falta de datos etiquetados disponibles; en este trabajo se observó un incremento
en el desempeño de clasificación de entre el 2\% y 3\% cuando se agregaron nuevos
datos cuya etiqueta fue asignada mediante algún algoritmo.



\bibliographystyle{splncs04}
\bibliography{referencias,referencias_cesar}
\end{document}